\documentclass[a4paper,11pt]{scrreprt}
\usepackage[utf8]{inputenc}
\usepackage[T1]{fontenc}
\usepackage[french]{babel}
\usepackage{graphicx}
\graphicspath{{./}}
\usepackage{booktabs}
\usepackage{amsmath}
\usepackage{siunitx}
\usepackage{circuitikz}
\usetikzlibrary{babel}

\title{Study on the experimental behaviour of a linear resistor, capacitor, and inductor alternative-current circuit}
\subtitle{Travaux pratiques génie électricité GIM 1 R1.04}
\author{Rong ZHOU \and Théo CUNY}
\newcommand{\univaddress}{IUT Hubert Curien, Université de Lorraine, 7 Rue Fusillés Résistance, 88000 Épinal}
\date{\today}
\newcommand{\contact}{ron@ronzz.org}

\begin{document}
\maketitle
\begin{center}
\univaddress\\
\contact
\end{center}

\section*{Introduction}
In this experiment, we aim to study the behaviour of a linear alternating-current circuit with components including resistors, capacitors, and inductors.

\section{Part 1 : study of phase shift caused by a single electrical component}

In this part, we connect our component of interest with a sinusoidal voltage source of a frequency of \SI{100}{Hz} with an auxiliary resistor, as shown in Figure~\ref{fig:1}.

\begin{figure}[h]
\centering
\begin{circuitikz}
	\draw
	(0,0) to[sV, v=$100\,\mathrm{Hz}$] (0,3)
				to[R=$10\,\Omega$] (3,3)
				to[european resistor, l_=Component] (3,0)
				-- (0,0);
\end{circuitikz}
\caption{Connection of the component with a sinusoidal voltage source and auxiliary resistor}
\label{fig:1}
\end{figure}

\begin{figure}[h]
\centering
\begin{circuitikz}
	\draw
	(0,0) to[sV, v=$100\,\mathrm{Hz}$] (0,3)
				to[R=$10\,\Omega$] (3,3)
				to[european resistor, l_=Component] (3,0)
				-- (0,0);

	% Oscilloscope connections
	\draw[-latex] (0,3) -- ++(0,0.5) node[above] {Ch1};
	\draw[-latex] (3,3) -- ++(0,0.5) node[above] {Ch2};
\end{circuitikz}
\caption{Oscilloscope connection: Channel 1 to voltage source, Channel 2 to auxiliary resistor}
\label{fig:2}
\end{figure}

Since a resistor causes zero phase shift, the phase shift between the voltage $U$ and the current $I$ of the circuit is equal to the phase shift between the source voltage $U$ and voltage across the resistor $U_{R}$.

The oscilloscope measures temporal shift $\Delta t$, from which we can mathematically calculate the phase shift $\Delta\phi$ with the formula
\[\Delta\phi = \frac{2\pi\,\Delta t}{T},\]
where $T$ is the period of the alternating voltage source.

\subsection{1.1 : Study of phase shift caused by a single resistor}

In this part, we use an auxiliary resistor of \SI{10}{\Omega} and the tested resistor of \SI{100}{\Omega} (Figure~\ref{fig:3}).

\begin{figure}[h]
\centering
\begin{circuitikz}
	\draw
	(0,0) to[sV, l=$100\,\mathrm{Hz}$] (0,3)
				to[R=$10\,\Omega$] (3,3)
				to[R=$100\,\Omega$] (3,0)
				-- (0,0);

% Oscilloscope connections
\draw[-latex] (0,3) -- ++(0,0.5) node[above] {Ch1};
\draw[-latex] (3,3) -- ++(0,0.5) node[above] {Ch2};

\end{circuitikz}
\caption{Circuit with sinusoidal source, auxiliary resistor, and 100~$\Omega$ resistor}
\label{fig:3}
\end{figure}

We observed zero temporal shift for channel 1 with regard to channel 2, from which we deduce zero phase shift, in correspondence with theoretical expectations.

\begin{figure}[h]
\centering
\includegraphics[width=0.7\textwidth]{oscilloscope-resistor.jpg}
\caption{Oscilloscope output showing zero phase shift for the resistor circuit}
\label{fig:oscilloscope-resistor}
\end{figure}

\subsection{1.2 : Study of phase shift caused by a single inductor}

\subsubsection{1.2.1 : Experimental determination}

In this part, we use an auxiliary resistor of \SI{100}{\Omega} with an inductor of \SI{1}{H} (Figure~\ref{fig:4}).

\begin{figure}[h]
\centering
\begin{circuitikz}
	\draw
	(0,0) to[sV, l=$100\,\mathrm{Hz}$] (0,3)
				to[R=$100\,\Omega$] (3,3)
				to[L=$1\,\mathrm{H}$] (3,0)
				-- (0,0);

% Oscilloscope connections
\draw[-latex] (0,3) -- ++(0,0.5) node[above] {Ch1};
\draw[-latex] (3,3) -- ++(0,0.5) node[above] {Ch2};

\end{circuitikz}
\caption{Circuit with sinusoidal source, auxiliary resistor, and 1~H inductor}
\label{fig:4}
\end{figure}

We observe a positive temporal shift of $\Delta t=\SI{2.5}{ms}$ for channel~1 with regard to channel~2, from which we deduce a positive phase shift
\[\Delta\phi=\frac{2\pi\,\Delta t}{T}=\frac{\pi}{2},\]
in correspondence with theoretical expectations for an ideal inductor at that frequency.

\begin{figure}[h]
\centering
\includegraphics[width=0.7\textwidth]{oscilloscope-inductor.jpg}
\caption{Oscilloscope output showing positive phase shift for the inductor circuit}
\label{fig:oscilloscope-inductor}
\end{figure}

\subsubsection{1.2.2 : Consideration of irrelevant variables}

We are conscious that the resistance of the auxiliary resistor and the frequency of the alternating voltage source may impact our results. We have therefore varied them to study the effect.

With $f=\SI{100}{Hz}$:
\begin{center}
\begin{tabular}{c|c|c|c}
 $R_{\mathrm{aux}}$ (\si{\Omega}) & 100 & 500 & 1000 \\
 \hline
 Experimental temporal shift (\si{ms}) & 2.25 & 1.5 & 1 \\
 Deduced phase shift & $\tfrac{\pi}{2}$ & $\tfrac{3\pi}{10}$ & $\tfrac{\pi}{5}$ \\
\end{tabular}
\end{center}

As the value of $R_{\mathrm{aux}}$ increases, the measured phase shift reduces. From this we can conclude that an additional resistor in series will change the behaviour of the circuit. It is therefore in the interest of experimental precision to take a small $R_{\mathrm{aux}}$.

With $R_{\mathrm{aux}}=\SI{1000}{\Omega}$, varying frequency:
\begin{center}
\begin{tabular}{c|c|c|c}
 $f$ (\si{Hz}) & 100 & 75 & 10 \\
 \hline
 $\Delta t$ (\si{ms}) & 2.25 & 3.75 & 10 \\
 Deduced phase shift & $0.5\,\pi$ & $0.56\,\pi$ & $0.1\,\pi$ \\
\end{tabular}
\end{center}

As the value of $f$ increases, the measured phase shift varies, which contradicts the simple theoretical expectation. We therefore conclude that frequency will impact the accuracy of our measurement. It is therefore preferable to take a reasonably large $f$, such as \SI{100}{Hz}, for experimental precision.

\subsection{1.3 : Study of phase shift caused by a single capacitor}

\subsubsection{1.3.1 : Experimental determination}

In this part, we use an auxiliary resistor of \SI{100}{\Omega} with a capacitor of \SI{1}{\mu F} (Figure~\ref{fig:5}).

\begin{figure}[h]
\centering
\begin{circuitikz}
	\draw
	(0,0) to[sV, l=$100\,\mathrm{Hz}$] (0,3)
				to[R=$100\,\Omega$] (3,3)
				to[C=$1\,\mu\mathrm{F}$] (3,0)
				-- (0,0);

% Oscilloscope connections
\draw[-latex] (0,3) -- ++(0,0.5) node[above] {Ch1};
\draw[-latex] (3,3) -- ++(0,0.5) node[above] {Ch2};

\end{circuitikz}
\caption{Circuit with sinusoidal source, auxiliary resistor, and 1~$\mu$F capacitor}
\label{fig:5}
\end{figure}

We observe a negative phase shift for the capacitor. Experimentally, a temporal shift of $\Delta t=\SI{1.5}{ms}$ (sign depending on channel reference) corresponds to a phase shift magnitude of $\tfrac{3\pi}{10}$, in partial correspondence with theoretical expectations for an ideal capacitor at that frequency. The difference in experimental observation and theoretical expectations may be explained by experimental imprecisions and imperfections in equipments used, such as an important resistance in the circuit.

\begin{figure}[h]
\centering
\includegraphics[width=0.7\textwidth]{oscilloscope-capacitor.jpg}
\caption{Oscilloscope output showing negative phase shift for the capacitor circuit}
\label{fig:oscilloscope-capacitor}
\end{figure}


\subsubsection{1.3.2 : Consideration of irrelevant variables}

Similar to section~1.2.2, we consider the irrelevant variables $R_{\mathrm{aux}}$ and $f$:

With $f=\SI{100}{Hz}$:
\begin{center}
\begin{tabular}{c|c|c|c}
 $R_{\mathrm{aux}}$ (\si{\Omega}) & 100 & 500 & 1000 \\
 \hline
 Experimental temporal shift (\si{ms}) & -2.5 & -2.2 & -1.6 \\
 Deduced phase shift & $-\tfrac{\pi}{2}$ & $-0.44\,\pi$ & $-0.32\,\pi$ \\
\end{tabular}
\end{center}

As the value of $R_{\mathrm{aux}}$ increases, the measured phase shift magnitude reduces. It is therefore preferable to take a small $R_{\mathrm{aux}}$ for better precision.

With $R_{\mathrm{aux}}=\SI{1000}{\Omega}$, varying frequency:
\begin{center}
\begin{tabular}{c|c|c|c}
 $f$ (\si{Hz}) & 100 & 75 & 10 \\
 \hline
 $\Delta t$ (\si{ms}) & 2.25 & 3.75 & 10 \\
 Deduced phase shift & $0.5\,\pi$ & $0.56\,\pi$ & $0.1\,\pi$ \\
\end{tabular}
\end{center}

Again, frequency impacts the measured phase shift; choose a sufficiently large and stable frequency for measurements (e.g. \SI{100}{Hz}).

\section{Part 2 : study of a RLC circuit in series under sinusoidal voltage}

\subsection{2.1 Experimental setup}

In this part, we connect a sinusoidal voltage source of a frequency of \SI{100}{Hz} with a resistor of \SI{20}{\Omega}, an inductor of \SI{0.2}{H} and a capacitor of \SI{3}{\mu F} in series. Channel~1 of the oscilloscope is connected to the resistor, and channel~2 to the voltage source (Figure~\ref{fig:6}).

\begin{figure}[h]
\centering
\begin{circuitikz}
	% Series RLC circuit
	\draw
	(0,0) to[sV, l=$100\,\mathrm{Hz}$] (0,3)
				to[R=$20\,\Omega$] (3,3)
				to[L=$0.2\,\mathrm{H}$] (6,3)
				to[C=$3\,\mu\mathrm{F}$] (9,3)
				-- (9,0) -- (0,0);

	% Oscilloscope connections
	\draw[-latex] (3,3) -- ++(0,0.5) node[above] {Ch1}; % Channel 1 to resistor
	\draw[-latex] (0,3) -- ++(0,0.5) node[above] {Ch2}; % Channel 2 to voltage source
\end{circuitikz}
\caption{Series RLC circuit with sinusoidal source, resistor, inductor, capacitor, and oscilloscope connections}
\label{fig:6}
\end{figure}

\subsection{2.2 Observations}

\paragraph{Effective voltages}
From the oscilloscope we read effective voltages $U$ of \SI{21}{V} from the voltage source, and $U_{R}$ of \SI{7.39}{V}. The period is \SI{10}{ms}, the frequency is \SI{100}{Hz}, which translates into an angular frequency of $\omega=2\pi f=200\pi\,\mathrm{rad\,s^{-1}}$.

\paragraph{Phase shift}
We observe a positive temporal shift of channel~1 with regard to channel~2 of $\Delta t=\SI{2.25}{ms}$, from which we deduce a phase shift
\[\Delta\phi = \frac{2\pi\,\Delta t}{T} \approx 0.45\,\pi.\]

\paragraph{Modeling}
If we model $U(t)=U_{m}\cos(\omega t)$, since $I_{m}=\dfrac{U_{Rm}}{R}$, so
\[I(t)=I_{m}\cos(\omega t-\tfrac{\pi}{2})=I_{m}\sin(\omega t)=\frac{U_{Rm}}{R}\sin(\omega t).\]

From $U_{R}$, we use Ohm's law $I=\dfrac{U_{R}}{R}$ to calculate the effective current
\[I_{\mathrm{eff}}=\frac{7.39}{20}=0.3695\:\mathrm{A}\approx\SI{369.5}{mA}.\]

\paragraph{Remarks}
With a phase shift $\Delta\phi\approx 0.45\,\pi$, this circuit behaves similarly to a predominantly inductive circuit, despite the presence of a resistor and a capacitor.

\begin{figure}[h]
\centering
\includegraphics[width=0.7\textwidth]{oscilloscope-RLC.jpg}
\caption{Oscilloscope output for the series RLC circuit}
\label{fig:oscilloscope-rlc}
\end{figure}

\section*{Conclusion}

In this experiment, we were partially able to experimentally verify the phase shifts produced by an inductor and a capacitor in an AC circuit (approximately $+\tfrac{\pi}{2}$ and $-\tfrac{\pi}{2}$ respectively for ideal components). We also studied a series RLC circuit to observe the combined effect of multiple components; phase contributions combine and the resulting phase depends on the relative magnitudes of the impedances rather than a simple arithmetic sum.

\end{document}

